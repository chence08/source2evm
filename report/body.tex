% Chapter 1

%\chapter{Chapter Title Here} % Main chapter title

\label{Chapter1} % For referencing the chapter elsewhere, use \ref{Chapter1} 

%----------------------------------------------------------------------------------------

% Define some commands to keep the formatting separated from the content 
\newcommand{\keyword}[1]{\textbf{#1}}
\newcommand{\tabhead}[1]{\textbf{#1}}
\newcommand{\code}[1]{\texttt{#1}}
\newcommand{\file}[1]{\texttt{\bfseries#1}}
\newcommand{\option}[1]{\texttt{\itshape#1}}

%----------------------------------------------------------------------------------------

\section{User-level Documentation}
In order to define the relation $\cdot[\cdot\leftarrow\cdot]\cdot$ we employ as usual an inductive definition using the following rules.\\\\
\begin{prooftree}
	\Infer0[for any name $v$]{v[v\leftarrow E_1]E_1}
\end{prooftree}\qquad
\begin{prooftree}
	\Infer0[for any name $x\neq v$]{x[v\leftarrow E_1]x}
\end{prooftree}\\\\
\begin{prooftree}
	\Hypo{E_1[v\leftarrow E]E_1'}
	\Hypo{E_2[v\leftarrow E]E_2'}
	\Infer2{E(E_2)[v\leftarrow E]E_1'(E_2')}
\end{prooftree}

\section{Developer-level Documentation}
In order to define the relation $\cdot[\cdot\leftarrow\cdot]\cdot$ we employ as usual an inductive definition using the following rules.\\\\
\begin{prooftree}
	\Infer0[for any name $v$]{v[v\leftarrow E_1]E_1}
\end{prooftree}\qquad
\begin{prooftree}
	\Infer0[for any name $x\neq v$]{x[v\leftarrow E_1]x}
\end{prooftree}\\\\
\begin{prooftree}
	\Hypo{E_1[v\leftarrow E]E_1'}
	\Hypo{E_2[v\leftarrow E]E_2'}
	\Infer2{E(E_2)[v\leftarrow E]E_1'(E_2')}
\end{prooftree}

